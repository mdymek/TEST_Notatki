\documentclass[a4paper]{article}

\usepackage{fullpage} % Package to use full page
\usepackage{parskip} % Package to tweak paragraph skipping
\usepackage{tikz} % Package for drawing
\usepackage{amsmath}
\usepackage{hyperref}
\usepackage[utf8]{inputenc}
\usepackage{lmodern}
\usepackage[MeX]{polski}
\usepackage[T1]{fontenc}
\usepackage{graphicx}
\usepackage{float}
\usepackage{subfiles}
\usepackage{booktabs}
\graphicspath{{graphics/}}
\usepackage{multirow}

\title{Notatki z kursu Testowanie oprogramowania}
\author{Małgorzata Dymek}
\date{2019/20, semestr zimowy}

\begin{document}
    \maketitle

    \section{Wprowadzenie.}
    \subfile{wyk1}

    \section{Testowanie w cyklu życia.}
    \subfile{wyk2}

    \section{Czarnoskrzynkowe techniki projektowania testów.}
    \subfile{wyk3}
    \subfile{wyk4}

    \section{Białoskrzynkowe techniki projektowania testów.}
    \subfile{wyk5}
    \subfile{wyk6}

    \section{Pozostałe techniki testowania.}
    \subfile{wyk7}

    \section{Testowanie niefunkcjonalne.}
    \subfile{wyk8}

    \section{Automatyzacja testowania.}
    \subfile{wyk9}

    \section{Zarządzanie testowaniem.}
    \subfile{wyk11}
    \subfile{wyk12}

    \section{Jakość oprogramowania.}
    \subfile{wyk13}
\end{document}
