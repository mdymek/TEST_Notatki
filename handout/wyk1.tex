\documentclass[../main.tex]{subfiles}

\begin{document}
    \subsection{Definicje.}

    \textbf{Pomyłka} - człowiek robi coś źle.

    \textbf{Defekt (usterka, bug, fault)} - \textbf{statyczny defekt} w kodzie (lub dokumentacji), skutek pomyłki człowieka.

    \textbf{Błąd} – \textbf{nieprawidłowy stan wewnętrzny} programu
    np. licznik pętli ustawiony na drugim zamiast pierwszym elemencie tablicy.

    \textbf{Awaria (failure)} – widoczne, nieprawidłowe działanie oprogramowania
    np. crash systemu, zwrócenie nieprawidłowego wyniku, komunikat o błędzie.

    \textbf{Incydent} – wydarzenie, które wymaga analizy

    \textbf{Suita testowa} - ????

    \textbf{Paradoks pestycydów} - jeśli ciągle uruchamiamy te same testy to tracą one zdolność do znajdywania nowych defektów.

    \textbf{Ewaluacja} - na początku i na końcu
    \begin{itemize}
        \item  \textbf{Walidacja} - dokonywana w celu
        potwierdzenia zgodności z założonymi celami użycia. (\textit{are we building the right thing?})

        \item \textbf{Weryfikacja} - sprawdzająca, czy produkt danej fazy spełnia wymagania
        (zwykle techniczne) ustalone podczas poprzedniej fazy. (\textit{are we building the thing right?})
    \end{itemize}

    \textbf{Testowanie to nie debugowanie}
    \begin{itemize}
        \item \textbf{Testowanie} znajduje awarie, sprawdza czy usterki zostały usunięte.
        \item \textbf{Debugowanie} lokalizuje miejsce usterki powodującej awarię i ją usuwa
    \end{itemize}

\end{document}