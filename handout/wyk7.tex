\documentclass[../main.tex]{subfiles}

\begin{document}

    \subsection{Testowanie eksploracyjne}

    \begin{table}[H]
        \begin{center}
            \begin{tabular}{p{8cm} p{8cm}}
                \begin{itemize}
                    \item \textbf{Każde podejście do testów jest w jakimś stopniu eksploracyjne.}

                    \item \textbf{Efektywność} techniki zależy od \textbf{stopnia posiadanych umiejętności}.

                    \item W eksploracji można używać \textbf{formalnych technik} projektowania testów!
                \end{itemize}
                &
                \textbf{Idea}:
                \begin{enumerate}
                    \item Pobieżna \textbf{eksploracja} systemu jako całości
                    \item \textbf{Sprawdzenie} modułów
                    \item Szczegółowa \textbf{analiza} dwóch rzeczy w jednym z modułów
                    \item Pogłębiona \textbf{analiza zagadnienia}
                \end{enumerate}
            \end{tabular}
        \end{center}
    \end{table}

    Forma działania: przez określony \textbf{czas}, \textbf{tester} wchodzi w interakcję z \textbf{produktem},
    aby wypełnić testerską \textbf{misję} i \textbf{zaraportować} wyniki.

    Przeprowadzane w \textbf{sesjach}, często z użyciem \textbf{karty testów} (test charter):
    \begin{itemize}
        \item np "zidentyfikuj i sprawdź
        wszystkie stwierdzenia z podręcznika użytkownika"
        \item \textbf{nie jest bardzo precyzyjna} – adresatem jest \textbf{tester dobrze znający system} i środowisko
        \item daje \textbf{swobodę} i \textbf{nie narzuca} konkretnych \textbf{rozwiązań} i podejść – wiele zależy od samego testera.
    \end{itemize}

    \begin{table}[H]
        \begin{center}
            \begin{tabular}{p{8cm} p{8cm}}
                \multicolumn{2}{c}{\textbf{Kiedy używać testowania eksploracyjnego?}}\\
                \multicolumn{2}{c}{Zwykle zawsze, gdy nie wiadomo jaki ma być następny test.}\\
                \begin{itemize}
                    \item gdy trzeba dać szybką informację zwrotną o nowym produkcie/cesze
                    \item gdy trzeba szybko nauczyć się nowego produktu
                    \item gdy użyto planowego testowania i szukamy różnorodności w testach
                    \item gdy chcemy znaleźć najważniejszy defekt w najkrótszym czasie
                \end{itemize}
                &
                \begin{itemize}
                    \item gdy chcemy sprawdzić pracę innego testera przez niezależną analizę
                    \item gdy chcemy zbadać i wyizolować konkretny defekt
                    \item gdy chcemy określić poziom konkretnego ryzyka w celu oceny zastosowania planowanego testowania w tym obszarze
                \end{itemize}
            \end{tabular}
        \end{center}
    \end{table}

    \subsection{Pomniejsze techniki}

    \begin{table}[H]
        \begin{center}
            \begin{tabular}{| p{8cm} | p{8cm} |}
                \hline
                \textbf{Zgadywanie błędów} & \textbf{Testowanie oparte na listach kontrolnych}\\
                \hline
                \begin{itemize}
                    \item czasem utożsamiane z \textbf{testowaniem ad-hoc}
                    \item technika niebazująca na żadnym systematycznym podejściu czy technice
                    \item opiera się na \textbf{doświadczeniu testera} w testowaniu
                \end{itemize}
                &
                \begin{itemize}
                    \item tester korzysta z \textbf{wysokopoziomowej listy elementów do
                    zanalizowania}, sprawdzenia lub zapamiętania
                    \item modelowa \textbf{lista kontrolna} może dotyczyć różnych aspektów, np.:
                    \begin{itemize}
                        \item charakterystyk jakościowych
                        \item standardów GUI
                        \item kluczowych operacji
                        \item standardów kodowania
                    \end{itemize}
                \end{itemize}\\
                \hline
                \hline
                \textbf{Ataki usterkowe} & \textbf{Techniki oparte na modelach defektów}\\
                \hline
                \begin{itemize}
                    \item idea: \textbf{atak} na oprogramowanie przez jego \textbf{interfejs}: użytkowy (GUI, API)
                    lub systemowy (system plikowy, interfejs bazodanowy, interfejs OS)
                \end{itemize}
                &
                \begin{itemize}
                    \item \textbf{taksonomia defektów} to system (hierarchicznych) kategorii
                    stworzony jako pomoc w klasyfikowaniu defektów
                \end{itemize}\\
                \hline
            \end{tabular}
        \end{center}
    \end{table}

    \subsection{Testowanie mutacyjne}
    \begin{table}[H]
        \begin{center}
            \begin{tabular}{p{8cm} p{8cm}}
                \begin{itemize}
                    \item metoda \textbf{oparta na składni}
                    \item można ją zaklasyfikować jako \textbf{białoskrzynkową}
                    \item uważana za jedną z \textbf{najmocniejszych} technik testowania
                \end{itemize}
                &
                \begin{itemize}
                    \item bezpośrednio \textbf{testuje testy}, nie program

                    \item \textbf{mutant} = zmodyfikowany, kompilowalny kod
                    \item \textbf{idea: każdy mutant powinien zostać zabity} przez
                    przynajmniej 1 test
                    \item \textbf{pokrycie mutacyjne} = $\frac{\# \text{zabitych mutantów}}{\# \text{mutantów}}$
                \end{itemize}
            \end{tabular}
        \end{center}
    \end{table}

    \begin{table}[H]
        \begin{center}
            \begin{tabular}{p{8cm} p{8cm}}
                \textbf{Zalety}
                \begin{itemize}
                    \item \textbf{wysoce zautomatyzowany} proces, wsparcie narzędziowe
                    \item \textbf{tani} we wdrożeniu
                    \item jedna z \textbf{najefektywniejszych} technik testowania
                \end{itemize}
                &
                \textbf{Wady}
                \begin{itemize}
                    \item \textbf{wymaga wielu zasobów, czasochłonny}
                    \item efektywność zależy od doboru operatorów mutacyjnych
                    \item problem \textbf{mutantów równoważnych} = mających ten sam efekt, np != i < w warunku pętli
                \end{itemize}
            \end{tabular}
        \end{center}
    \end{table}


    \subsection{Analiza statyczna}

    \begin{table}[H]
        \begin{center}
            \begin{tabular}{p{8cm} p{8cm}}
                \textbf{Cechy analizy statycznej} & \textbf{Typowe defekty wykrywane przez analizę statyczną}\\
                \begin{itemize}
                    \item \textbf{wczesne wykrywanie} defektów
                    \item identyfikacja defektów trudnych do wykrycia w testach
                    \item \textbf{wykrywanie zależności i niespójności} w modelach oprogramowania
                    \item zwiększenie \textbf{pielęgnowalności} kodu
                    \item \textbf{zapobieganie defektom}
                \end{itemize}
                &
                \begin{itemize}
                    \item odwołania do niezainicjalizowanej zmiennej
                    \item niespójne interfejsy,
                    \item niewykorzystane zmienne,
                    \item martwy kod,
                    \item brakująca lub błędna logika,
                    \item zbyt skomplikowane instrukcje,
                    \item naruszenie standardów kodowania,
                    \item słabe punkty zabezpieczeń,
                    \item naruszenie reguł modelowania, itp.
                \end{itemize}\\
            \end{tabular}
        \end{center}
    \end{table}

    \subsubsection{Techniki analizy statycznej}
    \begin{itemize}
        \item \textbf{Analiza złożoności}\\
        \textbf{Złożoność cyklomatyczna} (CC, McCabe cyclomatic complexity) to miara stopnia \textbf{skomplikowania struktury kodu}.

        Metody (równoważne) obliczenia CC:
        \begin{enumerate}
            \item CC = max liczba ścieżek liniowo niezależnych
            \item CC = E – N + 2
            \item CC = liczba zamkniętych obszarów CFG + 1
            \item CC = liczba decyzji w CFG + 1
            (switch z n>2 liczony jako n–1 decyzji)
        \end{enumerate}
        Generalnie, im mniejsza wartość CC, tym lepiej.

        \item \textbf{Parsowanie kodu} - analiza błędów poprawnych syntaktycznie.
        \item \textbf{Analiza przepływu danych}, np:
        \begin{itemize}
            \item przypisanie nieprawidłowej wartości do zmiennej (typy)
            \item użycie zmiennej przed jej zdefiniowaniem
            \item użycie usuniętej uprzednio zmiennej
            \item redefiniowanie zmiennej przed jej użyciem
        \end{itemize}


        \item \textbf{Graf wywołań}
        \begin{itemize}
            \item \textbf{Graf skierowany}, w którym:
            \begin{itemize}
                \item \textbf{wierzchołki} = jednostki oprogramowania (np. moduły, funkcje itp.)
                \item \textbf{krawędzie} = komunikacja między jednostkami (np. wywołanie)
            \end{itemize}

            \item \textbf{Złożoność} grafu wywołań: 1 za ifa, n-1 za switch, 2 za pętle.

            \item \textbf{Służy do:}
            \begin{itemize}
                \item określenia kolejności testów integracyjnych
                \item dolnego oszacowania na liczbę testów integracyjnych
            \end{itemize}

            \item \textbf{W testach integracyjnych} chcemy określić ich kolejność tak, aby:
            \begin{itemize}
                \item nie budować zbyt wielu namiastek/sterownków
                \item gdy nastąpi awaria, łatwo będzie zidentyfikować miejsce defektu
                \item będziemy testować rzeczywiste interakcje (np. faktyczne wywołania)
            \end{itemize}
        \end{itemize}
    \end{itemize}

    \subsection{Analiza dynamiczna}
    \begin{itemize}
        \item wykorzystywana do \textbf{wykrywania awarii}, gdzie symptomy mogą nie być natychmiastowo widoczne
        \item analiza dynamiczna \textbf{działa na uruchomionym programie}/systemie
    \end{itemize}

    \subsubsection{Przykładowe techniki analizy dynamicznej}
    \begin{itemize}
        \item \textbf{wykrywanie wycieków pamięci}
        \begin{itemize}
            \item symptomy: stopniowe pogarszanie się wydajności aplikacji
        \end{itemize}
        \item \textbf{wykrywanie dzikich wskaźników}
        \item \textbf{analiza wydajności}
        \begin{itemize}
            \item narzędzia mogą pomóc wykryć \textbf{wąskie gardła} wydajności
            \item np. informacja o \textbf{ilości wywołań} modułu podczas wykonania
            \item często wywoływane moduły to kandydaci do usprawnienia wydajności
            \item połączona informacja o dynamicznym zachowaniu systemu i o grafach
            wywołań pozwala testerowi zidentyfikować moduły mogące być kandydatami na szczegółowe testowanie
        \end{itemize}

        \item \textbf{analiza zachowania sieci}
        \item \textbf{analiza aplikacji przy użyciu profilera}
    \end{itemize}
\end{document}