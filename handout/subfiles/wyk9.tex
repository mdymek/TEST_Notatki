%! suppress = LineBreak
%! suppress = MissingLabel
%! suppress = FileNotFound
\documentclass[../main.tex]{subfiles}

\begin{document}
    \textbf{Efekt próbnika (probe effect)} - niezamierzony wpływ na zachowanie systemu spowodowany pomiarami tego systemu.
    Narzędzie które wpływa na wynik testu nazywamy \textbf{inwazyjnym}.


    \textbf{Zalecenia dotyczące architektury} systemu automatyzacji:
    \begin{itemize}
        \item \textbf{pojedyncza odpowiedzialność} komponentów; każdy odpowiada za pojedynczy obszar
        \item \textbf{rozszerzalność} - komponenty rozszerzalne, ale nie modyfikowalne
        \item \textbf{zastępowalność}
        \item \textbf{rozdzielność}
        \item \textbf{abstrakcyjność zależności} - komponenty zależą od abstrakcji
    \end{itemize}

    \subsection{Techniki automatyzacji testów}
    \begin{tabular}{| p{5cm} || p {5cm} | p{5cm} |}
        \hline
        \textbf{Technika} & \textbf{Zalety} & \textbf{Wady} \\
        \hline
        \hline
        \textbf{nagraj i odtwórz} &
        stosowalne na poziomie GUI lub API, łatwe do konfiguracji i użycia, nie
        wymaga znajomości języków
        &
        trudne w utrzymaniu, problemy gdy potrzeba czasu na odpowiedź systemu \\
        \hline
        \textbf{skrypty linearne} &
        brak żmudnych i kosztownych przygotowań, znajomość programowania niekonieczna gdy
        skrypt tworzony automatycznie
        &
        koszt automatyzacji liniowy ze względu na liczbę skryptów; trudne i kosztowne w utrzymaniu \\
        \hline
        \textbf{skrypty zorganizowane} &
        redukcja kosztów utrzymania, zmniejszenie kosztu automatyzacji nowych testów
        &
        zwiększone koszty początkowe tworzenia reużywalnych skryptów; wymaga umiejętności
        programowania \\
        \hline
        \textbf{data-driven testing} &
        niski koszt dodania testu; nie wymaga znajomości programowania; tanie w utrzymaniu
        &
        ograniczona możliwość przeprowadzania testów negatywnych \\
        \hline
        \textbf{keyword-driven testing} &
        tanie w utrzymaniu; swoboda w tworzeniu testów
        &
        kosztowna implementacja słów kluczowych; trudność w doborze właściwych słów \\
        \hline
    \end{tabular}

    \subsection{Testowanie oparte na modelu (MBT)}

    \begin{itemize}
        \item \textbf{Podstawowa idea: ulepszyć jakość i efektywność} projektu i implementacji testów przez:
        \begin{itemize}
            \item projekt wyczerpującego modelu MBT, zwykle z użyciem narzędzi,
            \item model jako specyfikacja projektu testów, automatyczna generacja przypadków testowych,
        \end{itemize}
        \item \textbf{Rodzaje modeli}: strukturalne, behawioralne, danych (UML).
    \end{itemize}

    \begin{table}[H]
        \begin{center}
            \begin{tabular}{p{8cm} p{8cm}}
                \textbf{Efektywność} & \textbf{Wydajność} \\
                \begin{itemize}
                    \item modelowanie ułatwia \textbf{komunikację} z interesariuszami
                    \item \textbf{zrozumienie}
                    \item \textbf{łatwiejsze zaangażowanie} interesariuszy
                    \item \textbf{łatwa identyfikacja „problematycznych” części systemu}
                    \item \textbf{wczesna generacja i analiza przypadków testowych} - możliwe przed stworzeniem systemu
                \end{itemize}
                &
                \begin{itemize}
                    \item \textbf{wczesne unikanie defektów} - weryfikacja wymagań
                    \item \textbf{możliwe reużycie} artefaktów MBT
                    \item \textbf{automatyzacja} - np. generacja testaliów
                    \item \textbf{adaptacja do zmian} - różne suity testów mogą być generowane z tego samego modelu
                    \item \textbf{redukcja kosztów przy zmianie wymagań} - „single point of maintenance”
                \end{itemize} \\
            \end{tabular}
        \end{center}
    \end{table}

    \begin{table}[H]
        \begin{center}
            \begin{tabular}{| p{8cm} | p{8cm} |}
                \hline
                \multicolumn{2}{|c|}{\textbf{Kryteria wyboru testów}} \\
                \hline
                \textbf{Oparte na pokryciu} & \textbf{Inne} \\
                \hline
                \begin{itemize}
                    \item \textbf{Wymagania połączone z modelem} - elementy modelu są połączone z wybranymi
                    wymaganiami. Pełne pokrycie wymagań odpowiada zestawowi testów całkowicie pokrywających wybrany zbiór wymagań.

                    \item Pokrycie \textbf{Elementów modelu MBT} - bazuje na \textbf{wewnętrznej strukturze modelu}.

                    \item \textbf{Oparte na danych} - związane są z takimi technikami projektowania testów jak:
                    \begin{itemize}
                        \item podział na \textbf{klasy równoważności}
                        \item analiza \textbf{wartości} brzegowych
                        \item testy kombinatoryczne (np. pair-wise)
                    \end{itemize}
                \end{itemize}
                &
                \begin{itemize}
                    \item \textbf{Losowe} przechodzenie przez model, wszystkie przejścia są równo
                    prawdopodobne. \textbf{W podejściu stochastycznym} wybór oparty o \textbf{rozkład prawdopodobieństwa}. Model reprezentuje profil
                    użycia (\textbf{profil operacyjny}).
                    \item \textbf{Oparte na scenariuszu/wzorcu} - scenariuszem może być use case lub
                    scenariusz użycia; wzorzec = częściowo zdefiniowany scenariusz, który można
                    zastosować do modelu MBT aby wyprowadzić jeden lub wiele testów.
                    \item \textbf{Sterowane projektem} - podejście oparte na dodatkowej informacji projektowej.
                \end{itemize} \\
                \hline
            \end{tabular}
        \end{center}
    \end{table}
\end{document}