\documentclass[../main.tex]{subfiles}

\begin{document}
    \subsection{Definicje.}

    \begin{table}[H]
        \begin{center}
            \begin{tabular}{| p{8cm}| p{8cm}|}
                \hline
                \multicolumn{2}{|c|}{\multirow{2}{*}{\textbf{TESTOWANIE}}}\\
                \multicolumn{2}{|c|}{}\\
                \hline
                \hline
                \textbf{Definicja} & \textbf{Pojęcia}\\
                \hline
                \textbf{Definicja testowania jest niejednoznaczna:}
                \begin{itemize}
                    \item Testowanie to \textbf{wykonywanie oprogramowania z intencją wykrywania}
                    tkwiących w nim \textbf{błędów}.
                    \item Testowanie to krytyczne sprawdzanie, obserwacja i \textbf{ewaluacja jakości}
                    oprogramowania.
                    \item Testowanie to proces \textbf{analizowania} fragmentu oprogramowania w celu
                    wykrycia różnic pomiędzy istniejącymi a pożądanymi warunkami (czyli
                    defektów) oraz w celu oceny cech tego fragmentu oprogramowania [IEEE].
                \end{itemize}
                &
                \textbf{Pomyłka} - człowiek robi coś źle.

                \textbf{Defekt (usterka, bug, fault)} - \textbf{statyczny defekt} w kodzie (lub dokumentacji), skutek pomyłki człowieka.

                \textbf{Błąd} – \textbf{nieprawidłowy stan wewnętrzny} programu
                np. licznik pętli ustawiony na drugim zamiast pierwszym elemencie tablicy.

                \textbf{Awaria (failure)} – widoczne, nieprawidłowe działanie oprogramowania
                np. crash systemu, zwrócenie nieprawidłowego wyniku, komunikat o błędzie.

                \textbf{Incydent} – wydarzenie, które wymaga analizy

                \textbf{Suita testowa} - ????
                \\
                \hline
                \textbf{Rola testowania} & \textbf{Cele testowania}\\
                \hline
                \begin{itemize}
                    \item dobrze zaprojektowany, zdany test
                    redukuje poziom ryzyka
                    \item testowanie zwiększa przekonanie o
                    jakości jeśli znajduje mało defektów
                    lub nie znajduje ich w ogóle
                    \item \textbf{jakość systemu} wzrasta gdy defekty są naprawiane
                    \item wymagania kontraktowe i prawne, standardy przemysłowe
                    \item jedna z czynności \textbf{QA} - Quality Assurance.
                \end{itemize}
                &
                \begin{itemize}
                    \item znajdowanie defektów (np. testy jednostkowe)
                    \item uzyskanie pewności co do poziomu jakości (np. testy akceptacyjne)
                    \item dostarczenie informacji do podjęcia decyzii (np. ocena jakości systemu)
                    \item zapobieganie pojawiania się defektów (np. projektowanie testów we wczesnych fazach życia)
                \end{itemize}

                Decyzja o zakresie i ilości testów zależy od \textbf{ryzyka} (ryzyko techniczne, biznesowe i safety risk)
                oraz \textbf{ograniczeń projektowych} (czas, budżet).\\
                \hline
                \textbf{7 uniwersalnych zasad testowania.} &  \textbf{Normy i standardy związanie z testowaniem}\\
                \hline
                \begin{enumerate}
                    \item Testowanie ujawnia usterki
                    \item Testowanie gruntowne jest niewykonalne
                    \item Wczesne testowanie
                    \item Kumulowanie się błędów
                    \item Paradoks pestycydów
                    \item Testowanie zależy od kontekstu
                    \item Mylne przekonanie o braku błędów
                \end{enumerate}
                &
                \begin{itemize}
                    \item \textbf{IEEE 829} – dokumentacja testowa
                    \item \textbf{IEEE 1008} – standard dla testowania jednostkowego
                    \item \textbf{IEEE 1028} – standard dla przeglądów i audytów
                    \item \textbf{ISO 9126} – model jakości (stara)
                    \item \textbf{ISO/IEEE 25000} – model jakości (nowa)
                    \item \textbf{ISO/IEEE 29119} – Software Testing Standard
                \end{itemize}\\
                \hline
            \end{tabular}
        \end{center}
    \end{table}

    \begin{figure}[H]
        \includegraphics[width=\linewidth]{definicje.png}
    \end{figure}



    \begin{table}[H]
        \begin{center}
            \begin{tabular}{| p{8cm}| p{8cm}|}
                \hline
                \multicolumn{2}{|c|}{\textbf{Ewaluacja}}\\
                \hline
                na początku & na końcu\\
                \hline
                \textbf{Walidacja} - dokonywana w celu
                potwierdzenia zgodności z założonymi celami użycia. &
                \textbf{Weryfikacja} - sprawdzająca, czy produkt danej fazy spełnia wymagania
                (zwykle techniczne) ustalone podczas poprzedniej fazy.\\
                \hline
                are we building the right thing? & are we building the thing right?\\
                \hline
                \hline
                \multicolumn{2}{|c|}{\textbf{Testowanie to nie debugowanie}}\\
                \hline
                \textbf{Testowanie} & \textbf{Debugowanie}\\
                \hline
                \begin{itemize}
                    \item znajduje awarie
                    \item sprawdza, czy usterka została poprawnie usunięta
                \end{itemize}
                &
                Na podstawie informacji o awarii:
                \begin{itemize}
                    \item \textbf{lokalizuje miejsce usterki} powodującej tę awarię
                    \item usuwa (\textbf{naprawia}) usterkę
                \end{itemize}\\
                \hline
            \end{tabular}
        \end{center}
    \end{table}
\end{document}